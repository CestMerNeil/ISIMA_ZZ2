% Created 2023-01-08 dim. 16:03
% Intended LaTeX compiler: pdflatex
\documentclass[11pt]{article}
\usepackage[utf8]{inputenc}
\usepackage[T1]{fontenc}
\usepackage{graphicx}
\usepackage{grffile}
\usepackage{longtable}
\usepackage{wrapfig}
\usepackage{rotating}
\usepackage[normalem]{ulem}
\usepackage{amsmath}
\usepackage{textcomp}
\usepackage{amssymb}
\usepackage{capt-of}
\usepackage{hyperref}
\usepackage[usenames,dvipsnames]{color}
\usepackage{listings}
\usepackage[a4paper,left=2cm,right=2cm,top=2cm,bottom=2cm]{geometry}

\hypersetup{
 pdfauthor={cliquot},
 pdftitle={},
 pdfkeywords={},
 pdfsubject={},
 pdfcreator={Emacs 26.3 (Org mode 9.4.6)}, 
 pdflang={English}}



\usepackage{fancyhdr}
\pagestyle{fancy}
\renewcommand\headrulewidth{1pt}
\renewcommand\footrulewidth{1pt}
\fancyhead[L]{\bfseries Simulation}
\fancyhead[R]{\includegraphics[scale=0.05]{img/INPisima.png}}
\fancyfoot[L]{CLIQUOT Théo}
\fancyfoot[R]{\today}

\usepackage{sectsty}
\usepackage{xcolor}
\sectionfont{\color{cyan}}
\subsectionfont{\color{teal}}
\subsubsectionfont{\color{violet}}

\begin{document}


\begin{titlepage}
  Cliquot
  \begin{center}
       \vspace*{1cm}

  \color{red}
  {\huge \bfseries Partie Analyse: \\[0.8cm]}
  \color{black}
  \vspace{0.5cm}
  {\huge   ALQ TP}
      
       \vspace{8cm}
       
       \begin{figure}[h]
         \centering
         \includegraphics[width=1\linewidth]{img/etoile.jpg}
         \caption{exemple de course de soleil}
       \end{figure}

      
       \vfill
            
       ISIMA \hfill \today
            
     \end{center}
     
   \end{titlepage}
   

\newpage
\begin{center}
  \vspace{10cm}
  \tableofcontents
  \vspace{2cm}
\lstlistoflistings%
\listoftables
\listoffigures
\end{center}
\newpage

\section{Cadre du projet}
\label{sec:orga47d616}

Le projet de simulation multi-agents que l'on veut développer suit le principe de la bataille de soleil des jeux mario kart.

Plusieurs équipes de joueurs se battent sur une carte afin de récupérer des soleils placé aléatoirement.

Le but étant de garder plus de soleil que les équipes adverses avant la fin de la limite du temps.

Dès qu'un joueur possède un soleil, il peut le perdre à nouveau si il est touché par un objet, en effet sur la carte il y a les soleils initiaux, mais ils existent aussi des objets qui réapparaissent régulièrement octroyant à celui qui le récupère des avantages ou autres.

\section{Objectif du projet}
\label{sec:org83e8dd6}
Le but est de reproduire une version simplifié de la bataille de soleil en utilisant le c++ et le paradigme objet, pour cela on va limiter ce projet selon les
règles suivantes:

\subsection{Les règles retenus}
\label{sec:org0d02d91}

\begin{itemize}
\item La map consiste en une grille plus ou moins grande, et les joueurs se déplacent dessus avec des déplacements
\end{itemize}
verticaux ou horizontaux.
\begin{itemize}
\item Les objets blessent uniquement ceux de l'équipe adverse
\item Les objets se limite actuellement à 3, avec:
\begin{itemize}
\item La carapace rouge, un missile à tête chercheuse qui va blesser un adversaire
\item La banane, un objet que l'on place sur le terrain et qui blesse un adversaire si il roule dessus
\item Le champignon, il donne une augmentation de vitesse à l'utilisateur.
\end{itemize}
\end{itemize}

\section{analyse des besoins}
\label{sec:orgdca7906}

Nous voulons implémenter dans un premier temps une partie de bataille de soleil, nous verrons plus tard si on peut élargir cette idée en ajoutant plus d'objets, des modifications à la carte \ldots{}


Afin de réaliser ce projet, nous avons donc besoins d'implémenter plusieurs entités qui sont assez implicite:

\begin{itemize}
\item Une carte
\item Des joueurs
\item Des équipes
\item Des objets activables
\item et les soleils
\end{itemize}

Il va ensuite falloir que chacune de ces entités puissent interagir entre eux, pour cela plusieurs actions doivent
être implémentées :

Pour le joueur:
\begin{itemize}
\item Se déplacer sur la carte
\item Utiliser un objet
\end{itemize}

Pour la Carte
\begin{itemize}
\item Faire apparaître les soleils et les joueurs au départ
\item Faire apparaître des objets régulièrement
\item Définir le temps de jeu avant la fin de la partie
\end{itemize}

Pour les objets:
\begin{itemize}
\item Activer leur actions respectives.
\end{itemize}

Pour les équipes
\begin{itemize}
\item Calculer le score.
\end{itemize}



Il va aussi falloir implémenter une logique d'action afin que nos joueurs (ici nos agents) puissent choisir quoi faire en fonction de leur environnement.

On peut donc résumer notre projet actuelle comme ceci:
\begin{figure}
  \centering
  \includegraphics[width=.9\linewidth]{img/plantuml.png}
  \caption{Class UML de notre projet}
\end{figure}

\end{document}