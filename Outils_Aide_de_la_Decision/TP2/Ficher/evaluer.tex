%在菜单中,编译器选择XeLaTex
 
\documentclass[11pt]{ctexart}
\usepackage[top=2cm, bottom=2cm, left=2.5cm, right=2.5cm]{geometry} %定义页边距
\usepackage{algorithm}
\usepackage{algorithmicx}
\usepackage{algpseudocode}
\usepackage{amsmath} %数学公式
\usepackage[UTF8]{ctex} %输出中文
\floatname{algorithm}{Algorithm} %算法
\renewcommand{\algorithmicrequire}{\textbf{Input:}} %输入
\renewcommand{\algorithmicensure}{\textbf{Output:}} %输出
 
\begin{document}
\renewcommand{\thealgorithm}{} %这里用来定义算法1,算法2等
    \begin{algorithm}
        \caption{evaluer()} %标题
        \begin{algorithmic}[1] %每行显示行号,1表示每1行进行显示
            \Require Structure du graphe considere;
                    Structure de la solution;
            \For{$i = 1 \to Nombre de pieces \ast Nombre de machines$}
                \State $j \leftarrow s.vb[i]$
                \State $c[j] \leftarrow c[j] +1$
                \If {$c[j] \geq 1$}
                    \State $m \leftarrow s.t[c[j]]$
                    \If {$PlusGrandCout$}
                        \State $Enregistrements$
                    \EndIf
                \EndIf
                \State $mach \leftarrow g.mach[j][c[j]]$
                \If{$AuMoinsUnePieceEstPassee$}
                    \State $Enregistrements$
                    \If{$PlusGrandDate$}
                        \State $Enregistrements$
                    \EndIf
                \EndIf
            \EndFor
            \For{$i=1 \to NombreDePieces$}
                \If{$PlusGrandeDateFinale$}
                    \State $Enregistrements$
                \EndIf
            \EndFor
        \end{algorithmic}
    \end{algorithm}
\end{document}