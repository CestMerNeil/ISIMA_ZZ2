%在菜单中,编译器选择XeLaTex
 
\documentclass[11pt]{ctexart}
\usepackage[top=2cm, bottom=2cm, left=2.5cm, right=2.5cm]{geometry} %定义页边距
\usepackage{algorithm}
\usepackage{algorithmicx}
\usepackage{algpseudocode}
\usepackage{amsmath} %数学公式
\usepackage[UTF8]{ctex} %输出中文
\floatname{algorithm}{Algorithm} %算法
\renewcommand{\algorithmicrequire}{\textbf{Input:}} %输入
\renewcommand{\algorithmicensure}{\textbf{Output:}} %输出
 
\begin{document}
\renewcommand{\thealgorithm}{} %这里用来定义算法1,算法2等
    \begin{algorithm}
        \caption{GenererTourGeant} %标题
        \begin{algorithmic}[1] %每行显示行号,1表示每1行进行显示
            \Require Structure du graphe considere;
                    Structure de la solution;
            \For{$i \to inst.n$}
                \State $min.l[0] \leftarrow i$
            \EndFor
            \State $min.l[0] \leftarrow inst.n +1$
            \State $min.tailleL \leftarrow inst.n$
            \State $s.tourGeant[0] \leftarrow 0 $
            \State $min.i \leftarrow 0$
            \For{$i \to inst.n$}
                \State $genererMin$
                \State $which \leftarrow 1$
                \State $cont \leftarrow 1$
                \While{$cont$}
                    \State $r \leftarrow randum number entre 101$
                    \If{$r \leq 80$}
                        \State $s.tourGeant[i] \leftarrow min.min[which][0]$
                        \State $min.l[min.min[which][2] \leftarrow min.l[min.tailleL]$
                        \State $min.talleL --$
                        \State $cout \leftarrow 0$
                    \Else{}
                        \State $which ++$
                        \If{$which \geq min.nbVoisins$}
                            \State $s.tourGeant[i] \leftarrow min.min[which][0]$
                            \State $cout \leftarrow 0$
                        \EndIf
                    \EndIf
                \EndWhile
                \State $min.i \leftarrow s.tourGeant[i]$
            \EndFor
            \State $s.toutGeant[inst.n+1] \leftarrow 0$
        \end{algorithmic}
    \end{algorithm}
\end{document}