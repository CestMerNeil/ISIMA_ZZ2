%在菜单中,编译器选择XeLaTex
 
\documentclass[11pt]{ctexart}
\usepackage[top=2cm, bottom=2cm, left=2.5cm, right=2.5cm]{geometry} %定义页边距
\usepackage{algorithm}
\usepackage{algorithmicx}
\usepackage{algpseudocode}
\usepackage{amsmath} %数学公式
\usepackage[UTF8]{ctex} %输出中文
\floatname{algorithm}{Algorithm} %算法
\renewcommand{\algorithmicrequire}{\textbf{Input:}} %输入
\renewcommand{\algorithmicensure}{\textbf{Output:}} %输出
 
\begin{document}
\renewcommand{\thealgorithm}{} %这里用来定义算法1,算法2等
    \begin{algorithm}
        \caption{GenererTourGeant} %标题
        \begin{algorithmic}[1]
            \Require inst
            \Require s
            \For{$i=1 \to inst.n$}
                \State $min.l[i] = i$
            \EndFor
            \State $min.taille\_l = inst.n$
            \State $s.tour\_geant[0] = 0$
            \State $min.i = 0$
            \For{$i=1 \to inst.n$}
                \State $s.tour_geant[i] = which_min(inst.min)$
                \State $min.i = s.tour\_geant[i]$
            \EndFor
            \State $s.tour\_geant[inst.n+1]=0$
        \end{algorithmic}
    \end{algorithm}
\end{document}